\begin{theorem}
    对换改变 $ n $ 元排列的奇偶性
\end{theorem}

\begin{proof}
    对于任意一个排列
    \[
    a = a_{1} \dots a_{k}a_{k+1} \dots a_{n}
    \]
    设 $ \tau (a) = A $ 若交换其中相邻的两项 $ a_{k}, a_{k+1} $
    对于有序数对 $ a_{i}a_{j} $ 其中 $ i < k $ 或 $ i > k + 1 $ 并不改变其逆序性质.

    若 $ a_{k} < a_{k+1} $ 则 $ A+1 $ 若 $ a_{k} > a_{k+1} $ 则 $ A-1 $
    所以交换任意相邻两项改变逆序数的奇偶性.

    对于更一般的情况
    \[
    a = a_{1} \dots a_{i} \dots a_{j} \dots a_{n}
    \]
    假设 $ a_{i} $ 和 $ a_{j} $ 之间不包括端点有 $ s $ 项, 
    交换 $ a_{i} $ 和 $ a_{i+1} $ 奇偶性改变一次, 只看$ a_{i} $ 和 $ a_{j} $ 之间的项
    \[
    a_{i}a_{i+1}a_{i+2} \dots a_{j}
    \rightarrow
    a_{i+1}a_{i}a_{i+2} \dots a_{j}
    \]
    再交换 $ a_{i} $ 和 $ a_{i+2} $ 奇偶性改变两次, 以此类推
    直到
    \[
    a_{i+1} \dots a_{i+s}a_{i}a_{j}
    \]
    时, 奇偶性改变了 $ s $ 次

    再次交换 $ a_{i} $ 和 $ a_{j} $ 奇偶性改变了 $ s+1 $ 次, 此时
    \[
    a_{i+1} \dots a_{i+s}a_{j}a_{i}
    \]
    然后让 $ a_{j} $ 以此与 $ a_{i+s}, a_{i+s-1} \cdots a_{i+1} $ 一共 $ s $ 个项交换,
    使得奇偶性又改变了 $ s $ 次, 结合之前的 $ s+1 $ 次, 我们发现奇偶性一共改变了 $ 2s+1 $ 次.

    这就意味着, 对于任意 $ s $ 都会改变奇数次, 如果原来 $ A $ 是偶数, 就会变成奇数, 
    如果原来是奇数, 就会变成偶数. 
    
    因此对于一切 $ a $ 交换其任意两项, 都会改变逆序数的奇偶性.
\end{proof}

\begin{theorem}
    任一排列 $ b_{1}b_{2}\dots b_{n} $ 都可以通过 $ s $ 步骤变换成
    $ a_{1}a_{2}\dots a_{n} $
    其中
    $ a_{1} < a_{2} < \cdots < a_{n} $
    并且其奇偶性与 $ s $ 相同
\end{theorem}

\begin{theorem}
    行列式的行列互换, 行列式的值不变.
\end{theorem}

\begin{theorem}
    行列式的某行翻倍, 行列式的值也翻倍.
\end{theorem}

\begin{theorem}
    行列式的某行是两组数的和, 可以看作.
\end{theorem}

\begin{theorem}
    行列式两行互换, 行列式值取反.
\end{theorem}

\begin{theorem}
    行列式两行相同值为零.
\end{theorem}

\begin{theorem}
    行列式一行倍加到另一行, 行列式的值不变.
\end{theorem}

